\subsection*{Learning Objectives}
	Students need to be able to\ldots
		\begin{itemize}\it 
			\item Model with a system of ODEs. \\[-20pt]
			\item Analyze the stability of equilibrium points. \\[-20pt]
			\item Linearize a nonlinear problem around a point of interest.
		\end{itemize}


%	
\subsection*{Context}
	
In lecture we studied some examples  of linearizing and studying the stability of equilibrium points.
We do so again in this tutorial.

We go in a bit more detail to prove some properties of the Lotka-Volterra model, such as showing that the solutions form periodic orbits.



\subsection*{Resources for TAs}

The \verb|python| code is here: \href{https://utoronto.syzygy.ca/jupyter/user-redirect/git-pull?repo=https://github.com/bigfatbernie/IBLMathModeling&subPath=tutorials/tutorial6/tutorial6.ipynb}{\tt tutorial6.ipynb}


\subsection*{Before Tutorial}


Send an announcement to students letting them know that they will need to bring a laptop and will be using Jupyter Notebooks \url{https://utoronto.syzygy.ca}.


\subsection*{What to Do}
	
Introduce the learning objectives for the day's tutorial. \\

Problem \ref{q1} is quite long, so there might not be any time for problem \ref{q2}. That is ok. If that happens, tell the students that they should work on problem \ref{q2} on their own.


\begin{enumerate}[label=\ref{q1}.(\alph*)]
	\item Students should analyze and explain the meaning of each term on the RHS
	\item Students will probably be a little stumped here. Ask them, if they had to write a tweet (xeet?) describing each equilibrium solution, what would they write?
	\item When we do the Jacobian, we are linearizing the ODEs, so somethings to keep in mind:
	\begin{itemize}
		\item All eigenvalues have real parts < 0 : Attracting / Asymptotically stable equilibrium
		\item One eigenvalue has real part > 0 : Repelling / Asymptotically unstable equilibrium
		\item All eigenvalues have real parts = 0 : Can't tell! (like when the second derivative is 0, we can't tell if it's a max/min/none!)
	\end{itemize}

	
	\item Insist that students do the \verb|python| parts. Tell them to use previous \verb|python| codes (from homework/lecture/...) - for plotting and for numerical methods.
	
	The Euler and Runge-Kutta methods are described in the lecture slides.
	
\end{enumerate}







%\subsection*{Notes}
%
%	\begin{enumerate}
%		\item Note
%	\end{enumerate}
	
	

	
	
